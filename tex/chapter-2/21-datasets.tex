\section{Обзор наборов данных}
В \autoref{tab:datasets} перечисленны все основные наборы данных, которые упоминались в работах \cite{koInsiderThreatDetection2017} и \cite{homoliakInsightInsidersIT2019}.\\

\newcommand{\rowfonttype}{}% Current row font
\newcommand{\rowfont}[1]{% Set current row font
   \gdef\rowfonttype{#1}#1%
}

\begin{table}[ht]
\centering
\begin{tabular}{ |p{3cm}||p{2cm}|p{7cm}|p{2cm}|  }
 \hline
 \multicolumn{4}{|c|}{Наборы данных} \\
 \hline
 \rowfont{\scriptsize}%
 \textbf{Название набора данных} & \rowfont{\scriptsize}\textbf{Публичный доступ?} & \rowfont{\scriptsize}\textbf{Описание} & \rowfont{\scriptsize}\textbf{Присутствует вредоносное поведение?}\\
 \hline\hline
 \rowfont{\normalsize}%
 CERT (2014) \cite{glasserBridgingGapPragmatic2013} & \vfil\hfil\textbf{+} & Синтетические данные со сгенерированным поведением пользователей и различными инсайдерскими сценариями & \vfil\hfil\textbf{+}\\
 \hline
 Enron(2004) \cite{klimtEnronCorpusNew2004} & \vfil\hfil\textbf{+} & Реальные данные почтовой переписки топ-менеджмента компании Enron (1998-2002) & \vfil\hfil\textbf{--}\\
 \hline
 RUU (2011) \cite{salemModelingUserSearch2011} & \vfil\hfil\textbf{--
 } & Поведение пользователей Windows & \vfil\hfil\textbf{+}\\
 \hline
 TWOS(2004) \cite{harilalTWOSDatasetMalicious2017} & \vfil\hfil\textbf{--} & Результат мультиплеерной игры, воспроизводящей взаимодействия в реальных компаниях & \vfil\hfil\textbf{+}\\
 \hline
 Schonlau(2001) \cite{StatisticalMethodsComputer} & \vfil\hfil\textbf{+} & Команды Unix, введенные пользователями & \vfil\hfil\textbf{--}\\
 \hline
 WUIL(2004) \cite{caminaWindowsUsersIntruderSimulations2014} & \vfil\hfil\textbf{--} & Поведение пользователей Windows & \vfil\hfil\textbf{+}\\
 \hline
 Purdue(1998) \cite{UCIMachineLearning} & \vfil\hfil\textbf{--} & Команды Unix, введенные пользователями & \vfil\hfil\textbf{--}\\
 \hline
 Greenberg(1998) \cite{greenbergUsingUNIXCollected1988} & \vfil\hfil\textbf{--} & Команды Unix, введенные пользователями & \vfil\hfil\textbf{--}\\
 \hline
\end{tabular}
\caption{наборы данных}
\label{tab:datasets}
\end{table}

Как видно из данного перечня, наборов данных с публичным доступом очень мало. Также очень большой проблемой среди всех наборов данных является то, что практически отсутствуют наборы данных, полученные в реальных условиях. Это связано с ограничениями на конфиденциальность поведенческой информации корпоративных работников и чувствительностью данных для безопасности компаний. Все приведенные наборы данных являются либо синтетическими наборами, как в случае CERT, либо получены в ходе эксперимента, проводившегося в форме игры, как в случае наборов WUIL и RUU. Единственным исключением является набор данных Enron, который содержит в себе корпоративную переписку реальной существовавшей компании, но обстоятельства получения данного набора довольно уникальны.

\begin{itemize}
\item Набор данных \textbf{CERT} является синтетическим набором данных, в котором сгенерирована структура компании, поведение её пользователей и пять различных сценариев инсайдерских атак. Набор данных называется в честь создавшего его подразделения Carnegie Mellon University, занимающееся вопросами компьютерной безопасности,- CERT. Набор данных публичный \cite{InsiderThreatTest} и вся доступная информация по нему находится в специальных файлах внутри. В состав набора данных входят отдельные файлы со структурой компании, входом/выходом в систему, подключением/отключением внешних устройств, посещенными вебсайтами, отправленными и полученными электронными письмами и операциями с файлами. В более поздних версиях была добавлена психометрика сотрудников и контент электронных писем, вебсайтов и файлов. \\
Контент появился в третьей версии набора данных и генерировался методом ``мешок слов'' - генерация набора ключевых слов, которые соответствует тематической модели пользователя. Этот набор не имеет внутри себя никакого порядка. Однако, в последней шестой версии, генерация контента происходит цельными предложениями.\\
Набор данных предполагает следующие сценарии зловредного поведения:\\
\begin{enumerate}
	\item Пользователь, который ранее не использовал съемные диски и не работал в нерабочие часы, начинает вход в систему в необычное время с использованием съемного носителя. Пользователь загружает данные на wikileaks.org и вскоре после этого покидает организацию.\\
	\item Пользователь начинает просматривать веб-сайты для поиска вакансий. Прежде чем покинуть компанию, он использует флэш-накопитель, чтобы украсть данные.\\
	\item Системный администратор загружает кейлоггер и использует флэш-накопитель для передачи его на машину своего руководителя. На следующий день он использует собранные ключи для входа в систему в качестве своего руководителя, рассылает ложные электронные письма, вызывая панику в организации, после чего он немедленно покидает организацию.\\
	\item Пользователь входит систему под личиной другого пользователя, ищет интересные ему файлы и присылает на свою домашнюю электронную почту.\\
	\item Член группы, недавно проредившейся увольнениями, загружает документы на Dropbox, планируя использовать их для личной выгоды.\\
\end{enumerate}
На данный момент CERT - самый популярный набор данных, который используется в большинстве работ посвященных теме обнаружения инсайдеров и является де-факто бенчмарком в данной теме.\\

\item Набор данных \textbf{Enron} состоит из 500 тысяч реальных электронных писем собранных за период от 1998 до 2002 года от 150 пользователей, большинство из которых относились к руководящему звену компании Enron. Несмотря на то, что некоторая часть электронных писем была удалена, так как они содержали конфиденциальную информацию, данный набор данных очень интересен в силу того, что он является единственным публичным набором данных с реальной корпоративной перепиской. Набор данных Enron также широко распространен в работах посвященных анализу данных.\\
\item \textbf{The Wolf of SUTD (TWOS)} представляет собой результат многопользовательской игры, специально разработанной для воспроизведения работы в реальных компаниях при наличии возможных инсайдеров. В игре участвовало 24 пользователя, организованные в 6 команд. Некоторым участникам временно давались роли ``предателя'', которые получали доступ к чужому устройству на 90 минут. Игра проводилась в течение недели. Собраны пользовательские данные об использовании мыши, клавиатуры, сети и данные с логов системных вызовов. \\ 
\item Набор данных \textbf{Schonlau} содержит в себе набранные 50 пользователями команды Unix, около по 15 тысяч команд на каждого пользователя. ``Зловредные'' сессии сгенерированы с помощью случайного смешивания данных, собранных от других неизвестных пользователей.\\
\item \textbf{Are You You (RUU)} был собран, чтобы преодолеть один самых значимых недостатков Schonlau набора данных --- отсутствие ``настоящих'' злонамеренных действий. В наборе данных собрана поведенческая информация с персональных компьютеров 18 пользователей в течение четырех дней. Для сбора поведенческой информации был разработан программа для мониторинга для платформы Windows, которая отслеживала всю деятельность, связанную с реестром, созданием и уничтожением процессов, GUI окон, доступом к файлам, а также активностью DLL библиотек.\\ Далее проводился эксперимент, в котором 60 участников случайно делились на три группы. Для каждой группы был выдан собственный сценарий действия, который участники должны были выполнить за 15 минут неограниченного доступа к файловой системе. Участники ``зловредной'' группы должны были найти в системе данные, которые можно украсть.\\ 
\item \textbf{Windows Users Intruder Simulation (WUIL)} включает в себя поведенческие данные собранные от 76 пользователей. Зловредные сессии были симулированы с помощью batch скриптов.\\
\item \textbf{Greenberg} является одним из первых наборов данных о поведении аутентифицированных пользователях. Включает в себе данные историй командных оболочек csh 168 пользователей Unix. Пользователи в наборе данных делятся на четыре группы по уровню пользовательских знаний и навыков. \\
\item Набор данных \textbf{Purdue} включает в себя команды набранные в командной оболочке tcsh Unix, собранные из 8 пользовательских компьютеров в течение двух лет. \\
\end{itemize}