\chapter{Заключение}

В данной работе было изучено множество актуальных средств решения проблемы автоматического обнаружения инсайдерских угроз на основе анализа поведенческих данных. Также были рассмотрены различные наборы данные, содержащие в себе информацию о пользовательском поведении.\\

В результате обзора набора данных был взят набор CMU CERT версии 4.2. В качестве валидационнной метрики был выбран ROC AUC. По итогу обзора существующих решений, для последующей работы была выбрана модель состоящая из LSTM-кодировщика и CNN-классификатора.
\\
Для сравнения также был использован ряд классических классификаторов (логистическая регрессия, SVM, KMeans, случайный лес). Модель со сверточным классификатором показала наилучшую метрику AUC ROC \textbf{0.9566}.\\

Был предложен ряд улучшений для существующего решения, которое позволяет достичь большей валидационной метрики. Как показали эксперименты, для обучения CNN-классификатора очень критична проблема переобучения. Использование слоёв batch-нормализации, взвешенной функции потерь и Dropout-слоев позволяют с этим бороться, значительно ускоряя сходимость нейросети и улучшая качество предсказаний. Лушчее качество AUC ROC из всех экспериментов - \textbf{0.9574}\\

Добавление тем контента не улучшило качество классификации до AUC ROC \textbf{0.9021}. Это возможно в силу синтетической природы набора данных или недостаточной обобщающей способности классификатора.