
\chapter{Аннотация}

Целью данной работы является изучение средств для создания системы автоматического обнаружения инсайдерских угроз по поведенческой информации пользователей.\\
Рассмотрен ряд различных существующих решений, которые основаны на синтетическом наборе данных инсайдерских угроз CMU CERT. На основе анализа моделей, для разработки выбрана архитектура нейросетевой модели с LSTM-кодировщиком поведения и CNN-классификатора. Результаты нейросетевых моделей сравниваются с классическими алгоритмами машинного обучения.\\
В процессе работы были испробованы различные техники и изменения для улучшения качества предсказаний моделей такие как применение embedding-слоя в LSTM-кодировщике, взвешенная функция потерь, использование Batch Normalization, добавление дополнительных признаков, в том числе признаков о содержимом файлов.

\chapter{Введение}

Современные корпоративные сети выделяет важная особенность, которая заключается в их размере: зачастую она может составлять тысячи и десятки тысяч устройств. При этом действие пользователй распределяется между множеством устройств и одна проблема решается целыми командами пользователей. При этих условиях важной задачей является обеспечение информационной безопасности, выявление случаев некорректного, непрофессионального или нецелевого использования ресурсов, оценка характеристик функционирования корпоративной сети и параметров использования ресурсов.\\

Основной задачей обеспечения информационной безопасности является раннее обнаружение внутренних вторжений. Под \textbf{внутренними (инсайдерскими) вторжениями} понимаются вредоносные для организации угрозы, которые исходят от людей внутри организации, таких как работники, бывшие работники, подрядчики или деловые партнеры, имеющих доступ к информации о методах безопасности внутри организации, данных и компьютерных системах.\\

Данный тип атак следует отличать от внешних атак, произошедших в следствие утечки корпоративных авторизационных данных сотрудников, что позволило атакущему получить доступ к информационным ресурсам компании под личиной украденных аккаунтов сотрудников.\\

% TODO: cite
Разделяют инсайдеские угрозы на злонамеренные и случайные. В случае злонамеренной атаки пользователь осознанно нарушаут установленные корпоративные правила пользования ресурсами для своих личных интересов. Случайные нарушения происходят в следтсвие неосторожности или технической неподготовленностью пользователя. Наиболее распространёнными причинами данного типа атак является желание сократить работу или оказать помощь коллеге, который не имеет подходящих полномочий. \\

% TODO: Более 30\% времени работы отделов IT безопасности уходит на обнаружение следов уже случившихся внутренних вторжений [2,3].
Не все внутренний вторжения удается предотвратить. Это позволяет анализировать историю уже произошедших атак, чтобы предотвратить будущие атаки. Это необходимо, чтобы оценить понесенные убытки и найти уязвимости в системе безопасности, которые сделали эту атаку возможной. Значительная часть времени работы специалистов информационной безопасности тратится на обнаржение следов уже произошедших атак.\\

% TODO: ponemon report
Более половины инсайдерских атак приходится на компаний, которые имеют свыше пяти тысяч сотрудников. \cite{ponemonReport2020}
Чем крупнее организация, тем больше вероятность, что она станет целью инсайдерских атак. Это обстоятельство объясняет повышенный интерес к данной проблемы со стороны крупных компаний и правительственных организаций.\\

Обнаружение инсайдерских угроз является трудной задачей, в силу того, что инсайдер пользуется доверенным ему доступом, чтобы относительно легко обходить системы внешней защиты информации. Затрудняет задачу также и обилие источников внутренних угроз. Это могут быть как внутренние сотрудники, так и внешние (аутсорсеры), бизнес-партнеры, поставщики услуг, подрядчики и бывшие сотрудники. Довольно часто возникают ситуации, когда пользователи по ошибке имеют больше привелегий, чем это необходимо для работы. Например, во множестве компаний при смени роли пользователя, у него остаются права предыдущей роли. Или при увольнении сотрудника, у него остаётся доступ к информационным ресурсом, содержащие конфиденциальную информацию. Привелегии пользователя имеют тендецию накапливаться с течением времени \cite{exabeam}.\\

Дополнительной сложностью является также большое число целей инсайдерских атак. Это могут быть как финансовые отчёты, ранний доступ к которым дает возможность инсайдерской торговли ценными бумагами компании, клиентские данные и технические документы, которые могут быть проданы конкурентам. Ценные данные обычно дублируются сразу на множестве носителях для резервного копирования, составления отчётов, разработки, тестирования и т. д. Поэтому количество целей для инсайдерских атак в компании растет с большой скоростью.\\

Инсайдерские атаки можно разделить на несколько этапов:

\begin{itemize}
	\item Обычный рабочий процесс. На этом этапе сотрудник не представляет угрозы.\\
	\item Начало внутреннего вторжения. Наступает некоторый переломный момент, после которого пользователь принимает решение подготовить и предпринять инсайдерскую атаку. Это может быть как предложение от конкурирующей организации, так и из-за личных мотивов сотрудника, например месть.\\
	\item Фаза исследования. Пользователь начинает активно собирать ценную корпоративную информацию и данные о том, как к ней можно получить доступ, используя имеющиеся привелеги. Также пользователь ищет пути расширить свои привелегии для достижения своих целей.\\
	\item Сокрытие данных. На данном этапе пользователь исследует имеющиеся в организации системы информационной безопасности, чтобы найти наиболее безопасный способ эксфильтрации данных. Пользователь предпринимает попытки вывода информации за информационный периметр. Пользователь выполняет только те действия, которые в случае раскрытия можно было бы оправдать неосторожностью. Обычно это происходит с помощью фиктивных данных, которые по своей структуре похожи на данные, которые пользователь пытается украсть, но не содержат в себе чувствительной информации. Затем пользователь тестирует различные каналы вывода данных и методы их обработки, на которые бы системы информационной безопасности не сработали бы.\\
	\item Определившись с целевыми данными для кражи и методом её утечки, пользователь переходит к финальному этапу - эксфильтрации данных. \\
\end{itemize}

Важно отметить, что подготовительные этапы до эксфильтрации занимают недели и месяцы. При этом в это время пользователь проявляет поведение, которое до этого не было для него свойственно. Это выражается как и в операциях, которые совершает пользователь, так и в документах, с которыми он работает.\\

Существует ряд технологий, для решения проблем инсайдерских угроз:\\
\begin{itemize}
	\item DLP (Data Loss Prevention) – предотвращение утечек с помощью анализа потоков данных, пересекающих периметр информационной системы
	\item SIEM (Security Information and Event Management) – анализ в реальном времени событий безопасности
	\item IAM (Identity and Access Management) – управление учетными данными пользователей, системами контроля и управления доступом
	\item UEBA (User and Entity Behavior Analytics) – анализ действий пользователя и принятие решения на основе исторических данных
\end{itemize}

DLP, SIEM и IAM системы способны бороться с инсайдерами только на самом последнем этапе утечки данных, когда инсайдер пытается провести эксфильтрацию данных. Эти системы не учитывают предшествующий период аномального поведения пользователя. Поэтому существует отдельный класс решений UEBA, который направлен на анализ поведения пользователей и способны обнаруживать ранние признаки утечки информации.\\

UEBA-системы проводят мониторинг широкого спектра действий пользователя и принимают решение об аномальности его поведения на основе исторических данных о легитимной работе пользователя и обнаруженных атаках, проведенных ранее. Это сильно отличает UEBA-системы от, например, DLP системы, которые работают на основе сформулированных экспертами политик безопасности. Использование исторических данных дает UEBA-системам очень важное преимущество над остальными системами безоспасности, которое заключается в раннем обнаружении признаков утечек. Основная цель UEBA-систем состоит в предоставлении анализа отделу информационной безопасности организации с описанием пользователей, проявляющих аномальное поведение, и описанием, почему система посчитала поведение пользователя аномальным. По определению, приведенном в отчете  Gartner \cite{GartnerReportMarket2019}, UEBA-системы на основе методов машинного обучения выполняют построение и применение моделей поведения (профилей) пользователей для выявления признаков аномального поведения.\\

В основе анализа проводимого UEBA-систем лежат статистические модели и модели машинного обучения. Эти методы также могут комбинироваться с ручными правилами, составленными экспертами. Существует три варианта применения моделей машинного обучения:\\

\begin{itemize}
	\item Обучение с учителем (Supervised). В этом случае модели для обучения подается историческое поведение пользователя с разметкой, является ли это поведение инсайдерскими или нет. При обучении модель учится воспроизводить заданную разметку поведения на данных, которые модель ранее не видела. Большим затруднением данного варианта заключается в необходимости достаточного большого размеченного набора данных, в том числе и с примерами вредоносного поведения.\\
	\item Обучение без учителя (Unsupervised). В противовес предыдущему варианту, на обучение модели подаются данные без разметки. Данные системы не могут отличить вредоносное поведение от нормального --- они могут лишь указать на пользователей, чьё поведение отличается от нормального. Поэтому интерпретация выводов модели лежит на эксперте-аналитике.\\
	\item Обучение с подкреплением (Reinforced). В этом случае используется гибрид Unsupervised-модели. Результаты срабатываний передаются обратно модели, чтобы снизить количество ложных срабатываний и повысить точность предупреждений. Такой подход "обучения на лету" может занять довольно длительное время, прежде чем модель станет эффективной в обнаружении злонамеренного поведения.\\
\end{itemize}

Современные модели машинного обучения дают хорошие результаты во многих задачах благодаря использованию глубокого обучения. Суть глубокого обучения состоит в том, что модель учится на заданных ей тренировочных данных и начинает из них генерировать новые признаки, которые представляют наибольшую пользу для аналитики.

Эксперты Gartner \cite{GartnerReportMarket2019} отмечают, что все больше заказчиков систем информационной безопасности узнают о различных типах моделей аналитики и их нюансах. Это оказывает очень большое влияние на поставщиков, которые не инвестируют в более продвинутую аналитику, поскольку заказчики предпочитают поставщиков с более продвинутой аналитикой.\\

Одним из наиболее важных этапов построения любой системы мониторинга пользовательского поведения является выбор данных, которые будут подаваться на вход реализуемой системе. Можно выделить два основных типов данных: 
\begin{itemize}
	\item Контентная информация. Содержит в себе неструктурированные данные, с которыми работают пользователи. Обычно представляет собой содержимое документов, файлов и веб-страниц, ассоциированные с различными операциями пользователей.
	\item Контекстная (журналируемая) информация. Представляет собой структурированные данные, описывающие атрибуты операций, которые пользователи выполняют с документами. Этот тип данных собирается из системных журналов прикладных программ и операционной системы.\\
\end{itemize}

Под поведенческой информацией будем понимать совокупность этих двух типов данных.

\section*{Актуальность задачи}

По отчетам Ponemon за 2020 год \cite{PonemonReport20182018} стоимость ущерба от инсайдерских угроз составляет 11,45 миллионов долларов в среднем за атаку. При этом средняя стоимость атак выросла на 31\% в два раза с 2018 года. По тому же отчету, частота инсайдерских инцидентов также заметно выросла. Также приводится, что из всех методов борьбы с инсайдерской угрозой UEBA системы лидируют по степени сокращения финансового ущерба. В среднем они сокращают 3,42 миллиона долларов. Это как раз связано с тем, что UEBA системы способны обнаруживать аномальное поведение пользователя на ранних этапах утечки информации. Это подтверждается в \cite{2019InsiderThreat}, где подчеркивается, что в силу того, что инсайдеры имеют мало препятствий, время с первых несанкционированных действий до обнаружения часто занимает месяцы и годы.\\

Существующие UEBA-решения нацелены в первую очередь на анализ структурированной контекстной информации о поведении пользователя которые включают в себя обычно данные об операциях с файлами, электронной почты, подключении новых устройств и данные из системного журнала операционной системы. Это можно объяснить тем, что анализ контентной информации намного более затруднителен из-за остуствия структуры и очень большого объёма по сравнению с контекстной информацией.\\

Однако анализ контента позволяет выявлять специальные случаи, при которых изменение поведение пользователя проявляется лишь в изменение содержимых файлов, с которыми он работает. В случае анализа одной только контекстной информации, выявление таких случаев представляется затруднительным. Таким образом, на сегодняшний день не существуют разработанных методов обнаружения аномального поведения пользователей на основе анализа контентной информации с использованием методов машинного обучения. Это позволяет утверждать о ценности контента для рассматриваемой задачи. Актуальность анализа текстовых данных также подвтерждается отчётом Gartner \cite{GartnerReportMarket2019}.\\

Согласно McKinsey \cite{InsiderThreatHuman} из всех опубликованных в VERIS Community Database публичных случаях утечки информации за период с 2012 по 2017 год, в 50\% присутствовал элемент инсайдерской угрозы. При этом, 38\% случаев из этих наличествует злонамеренные действия инсайдеров. В отчете Verizon за 2019 год \cite{2019InsiderThreat} приводится, что 57\% взломов базы данных организации включают в себя инсайдерские угрозы.\\

Отчеты Haystax \cite{veriatoInsiderThreatReport}\cite{companyInsiderThreatReport} и Gartner \cite{EmergingInsiderThreat2018} подтверждают растущие опасения организаций по этому поводу и тренд на рост частоты атак инсайдеров.

% Согласно отчету Gartner \cite{GartnerReportMarket2019} рынок самостоятельных UEBA-систем умирает, но в данный момент очень востребованы SIEM-системы, которые включают в себя UEBA.\\

По приведенным выше отчетам становится очевидной актуальность изучения и разработки моделей машинного обучения, способные определять ранние признаки аномального поведения корпоративных пользователей.\\

\chapter{Постановка задачи}

\begin{enumerate}
	\item Исследование и разработка методов обнаружения аномального поведения пользователей на основе собираемой поведенческой информации следующих типов:
	\begin{enumerate}
		\item контекст пользовательских операций;
		\item контент файлов;
	\end{enumerate}
\end{enumerate}