
% \chapter{Аннотация}

% Целью данной работы является изучение средств для создания системы автоматического обнаружения инсайдерских угроз по поведенческой информации пользователей.\\
% Рассмотрен ряд различных существующих решений, которые основаны на синтетическом наборе данных инсайдерских угроз CMU CERT. На основе анализа моделей, для разработки выбрана архитектура нейросетевой модели с LSTM-кодировщиком поведения и CNN-классификатора. Результаты нейросетевых моделей сравниваются с классическими алгоритмами машинного обучения.\\
% В процессе работы были испробованы различные техники и изменения для улучшения качества предсказаний моделей такие как применение embedding-слоя в LSTM-кодировщике, взвешенная функция потерь, использование Batch Normalization, добавление дополнительных признаков, в том числе признаков о содержимом файлов.

\chapter{Введение}

\textbf{Внутренние (инсайдерские) вторжения} — это вредоносные для организации угрозы, которые исходят от людей внутри организации, таких как работники, бывшие работники, подрядчики или деловые партнеры, имеющих доступ к информации о методах безопасности внутри организации, данных и компьютерных системах.\\

Инсайдерские угрозы в данный момент является серьезной проблемой для многих организаций. Их обнаружение является очень трудной задачей в силу того, что инсайдеры для злонамеренных действий используют доверенный им доступ, из-за чего они могут легко обходить системы внешней защиты информации.\\

Существует ряд технологий, для решения проблем инсайдерских угроз:\\
\begin{itemize}
\item DLP (Data Loss Prevention) – предотвращение утечек с помощью анализа потоков данных, пересекающих периметр информационной системы\\
\item SIEM (Security Information and Event Management) – анализ в реальном времени событий безопасности\\
\item IAM (Identity and Access Management) – управление учетными данными пользователей, системами контроля и управления доступом\\
\item UEBA (User and Entity Behavior Analytics) – анализ действий пользователя и принятие решения на основе исторических данных\\
\end{itemize}

DLP, SIEM и IAM системы способны бороться с инсайдерами только на самом последнем этапе утечки данных, когда инсайдер пытается провести эксфильтрацию данных. При этом, до этого момента инсайдер проходит стадии подготовки утечки, которая может длиться месяцами и в течение этого этапа инсайдер проявляет аномальное поведение. Поэтому существует отдельный класс решений UEBA, который направлен на анализ поведения пользователей и способны обнаруживать ранние признаки утечки информации.\\

Под поведенческой информацией понимается совокупность данных, которые описывают:\\
\begin{itemize}
\item Контекст – структурированные данные, описывающие атрибуты операций, которые пользователь выполняет с документами;
\item Контент – неструктурированные данные в виде содержимого документов, ассоциированные с операциями пользователя.
\end{itemize}

\section*{Актуальность задачи}

По отчетам Ponemon за 2020 год \cite{PonemonReport20182018} стоимость ущерба от инсайдерских угроз составляет 11,45 миллионов долларов в среднем за атаку. При этом средняя стоимость атак выросла на 31\% в два раза с 2018 года. По тому же отчету, частота инсайдерских инцидентов также заметно выросла. Также приводится, что из всех методов борьбы с инсайдерской угрозой UEBA системы лидируют по степени сокращения финансового ущерба. В среднем они сокращают 3,42 миллиона долларов. Это как раз связано с тем, что UEBA системы способны обнаруживать аномальное поведение пользователя на ранних этапах утечки информации. Это подтверждается в \cite{2019InsiderThreat}, где подчеркивается, что в силу того, что инсайдеры имеют мало препятствий, время с первых несанкционированных действий до обнаружения часто занимает месяцы и годы.\\ \\

Согласно McKinsey \cite{InsiderThreatHuman} из всех опубликованных в VERIS Community Database публичных случаях утечки информации за период с 2012 по 2017 год, в 50\% присутствовал элемент инсайдерской угрозы. При этом, 38\% случаев из этих наличествует злонамеренные действия инсайдеров. В отчете Verizon за 2019 год \cite{2019InsiderThreat} приводится, что 57\% взломов базы данных организации включают в себя инсайдерские угрозы.\\

Отчеты Haystax \cite{veriatoInsiderThreatReport}\cite{companyInsiderThreatReport} и Gartner \cite{EmergingInsiderThreat2018} подтверждают растущие опасения организаций по этому поводу и тренд на рост частоты атак инсайдеров.

Согласно отчету Gartner \cite{GartnerReportMarket2019} рынок самостоятельных UEBA-систем умирает, но в данный момент очень востребованы SIEM-системы, которые включают в себя UEBA.\\

По приведенным выше отчетам становится очевидной актуальность изучения и разработки моделей машинного обучения, способные определять ранние признаки аномального поведения корпоративных пользователей.

Существующие UEBA-решения нацелены в первую очередь на анализ структурированной контекстной информации о поведении пользователя которые включают в себя обычно данные об операциях с файлами, электронной почты, подключении новых устройств и данные из системного журнала ОС. Это объясняется тем, что анализ контентной информации намного более затруднителен из-за своей неструктурированности и очень большого объёма. В то же время, анализ контента позволяет выявлять случаи, при которых поведения пользователя остаётся таким же, но меняется содержимое файлов, с которыми он работает. Анализ только структурированной информации такие случаи выявить не может, что позволяет утверждать о ценности контента для рассматриваемой задачи. Актуальность анализа текстовых данных также подвтерждается отчётом Gartner \cite{GartnerReportMarket2019}.\\

Рассмотрим существующие методы построения моделей машинного обучения для задачи автоматического обнаружения инсайдерских угроз на основе поведенческих данных. Также рассмотрим доступные публичные наборы данных, которые потенциально могут быть использованы для обучения моделей.

\section{Обзор моделей}

Синтетический набор данных разработанный подразделением CERT Carnegie Melon University пользуется очень большой популярностью в исследованиях на данную тему. Поэтому, по умолчанию, во всех приведенных статьях ниже используется набор данных CERT версии 4.2. Также по умолчанию содержимое писем, файлов и вебсайтов из данного набора данных не используется.\\

В работе \cite{rashidNewTakeDetecting2016} используется скрытая марковская модель (Hidden Markov Model - \textbf{HMM}). Скрытая марковская модель представляет собой обычную марковскую модель, в которой модель выводит некоторый символ каждый раз перед переходом в следующее состояние. Модель называется скрытой, поскольку мы наблюдаем только выходы модели, а последовательность переходов состояний скрыта от нас.\\
В данной работе все действия пользователей кодируются числами. Затем действия собираются отдельно для каждого пользователя и сортируются по времени. На первых пяти неделях модель обучается (используется предположение, что за это время инсайдерского поведения не было), для последующих недель модель предсказывает вероятность аномальности поведения и только потом обучается. Поведение за неделю считается аномальным, если оно превышает заданный порог.\\
Эксперименты на CERT 4.2 после подбора гиперпараметров показали AUC ROC $0.83$.\\

В \cite{aldairiTrustAwareUnsupervised2019} также применяется метод обучения без учителя и задача обнаружения инсайдерских угроз ставится как задача обнаружения аномалий. В этой работе сравниваются два классических метода нахождения аномалий:\\
\begin{enumerate}
\item Isolation Forest (Изолирующий лес) - разновидность алгоритма случайного леса, в котором строятся случайные бинарные решающие деревья. Аномальными объектами считаются те, которые часто оказываются в листьях с низкой глубиной
\item Метод опорных векторов для одного класса (One Class SVM) представляет собой обычный SVM, в котором все данные имеют положительную метку. При оптимизации модель пытается вместить как можно больше объектов в как можно меньшую гиперплоскость. В результате получается граница, по одну сторону которой максимально плотно упакованы наблюдения из тренировочной выборке, а по другую будут находиться аномальные значения.\\
\end{enumerate}
Работа проводилась на наборе данных CERT. Моделям на вход подавались данные агрегированные по разным временным периодам - дням, месяцам, полугодиям и годам. Также в качестве дополнительного признака используется trust score (показатель доверия), который означает оценку аномальности пользователя для предыдущего периода. Авторы показывают, что этот признак заметно улучшает точность предсказаний моделей, особенно в случаях моделей, в которых данные агрегируются по малым периодам.\\

В статье \cite{luInsiderThreatDetection2019} применяется рекуррентная нейронная сеть LSTM для поиска аномалий в поведении пользователей.\\
%%% Описание LSTM %%%
\textbf{Long Short Term Memory (LSTM)} - разновидность рекуррентных нейронных сетей. Обучение рекуррентных сетей является трудной задачей из-за проблемы взрывающихся и затухающих градиентов. Более того, RNN не могут удерживать в ``памяти'' долговременные зависимости. Чтобы решить эту проблему были предложены LSTM сети, которые способны удерживать информацию в течение длительного времени. LSTM имеет три фильтра, которые помогают защищать и контролировать состояние ячейки. Фильтр входа (input gate) решает, какие значения следует обновить при подаче новой информации. Фильтр ``забывания'' (forget gate) определяет какую часть значений следует ``забыть'' и фильтр выхода (output gate) определяет какую часть значения можно вывести в качество выхода.\\
Шаг обновления LSTM:\\
\begin{align}
i_t = &\: \sigma(\mathbf{W_i}e_t+\mathbf{U_i}h_{t-1}+b_i)\\
f_t = &\: \sigma(\mathbf{W_f}e_t+\mathbf{U_f}h_{t-1}+b_f)\\
o_t = &\: \sigma(\mathbf{W_o}e_t+\mathbf{U_o}h_{t-1}+b_o)\\
g_t = &\: \tanh(\mathbf{W_g}e_t+\mathbf{U_g}h_{t-1}+b_g)\\
c_t = &\: f_t\odot c{t-1}+i_t\odot g_t\\
h_t = &\: o_t \odot \tanh(c_t)
\end{align}
где $\mathbf{W}$, $\mathbf{U}$ - обучаемые матрицы параметров, $e_t$ - входной вектор, $h_t$ - выходной вектор, $c_t$ - вектор состояния сети, $f_t$, $i_t$, $o_t$ - векторы вентилей.\\
В работе \cite{luInsiderThreatDetection2019} Каждый отдельный тип действия пользователей кодируется определенным числом. Последовательность закодированных действий разбивается на пересекающиеся блоки размера $n$ с помощью скользящего окна. Размер блоков является гиперпараметром. Далее последовательность блоков действий подается н вход LSTM-сети, которая пытается предсказать следующее действие пользователя. Предполагается, что при нормальном поведении пользователя, правильное действие пользователя попадает в $g$ наиболее вероятных предсказаний. $g$ также является гиперпараметром.\\
В работе проведены эксперименты, которые показывают, что LSTM действительно обучается паттернам поведения пользователей и находить аномальное поведение. Наилучший результат при подборе гиперпараметров: точность - $0.84$ и полнота - $0.60$.\\
В статье приводится предположение, что некоторые случаи, в которых модель ошибочно не распознала аномальное поведение, можно разрешить с помощью анализа контента.

В \cite{noeverClassifierSuitesInsider2019} было испробовано множество различных семейств алгоритмов машинного обучения. В этой статье приводится к заключению, что алгоритмы случайного леса и бустинга показывают наилучшие результаты. Важно отметить, что они в качестве признаков также использовали сентимент-анализ текста электронной почты и содержания посещенных сайтов.

При применении моделей машинного обучения качество моделей очень сильно зависит от признаков, которые были вручную сгенерированы исследователями. Однако в последнее время наблюдается очень большой интерес к нейронным сетям, в том числе из-за того, что они способны автоматически выучивать хорошее признаковое представление данных. Поэтому и в данной теме множество последних работ использует нейросети.\\

В работе \cite{brdiczkaProactiveInsiderThreat2012b} используется информация о социальном графе для нахождения аномалий в нем и поведенческая информация пользователей для психологического профилирования. Интересно отметить, что в данной работе в качестве данных использовались данные из популярной онлайн многопользовательской игры World of Warcraft(WoW). Они собрали данные о социальном графе игроков внутри игры и его изменениях за шесть месяцев. Авторы с помощью своего методы пытались предсказать то, что игрок в скором времени покинет гильдию (социальную группу внутри игры). В пользу необычного выбора набора данных приводится, что данных много, содержат в себе зловредные поведения, публичны и не ограничены правилами конфиденциальности. Авторы утверждают, что предложенный ими метод можно применять также и на реальных предприятиях.\\

В статье \cite{yuanInsiderThreatDetection2018b} используется следующий подход: сначала LSTM выучивает поведение пользователя по его действиям и извлекает временные признаки, затем извлеченные признаки подаются на вход CNN классификатору.\\

%%% Описание CNN %%%
\textbf{Сверточные нейронные сети CNN} - специальная разновидность архитектур нейронных сетей, в которой слои, выполняющие свертку, чередуются со слоями субдискретизации. На данный момент, CNN является одним из лучших алгоритмов по распознаванию и классификации изображений.\\

Поведения пользователя рассматривается как последовательность действий и действия одного пользователя соответствует одному "предложению", как в обычных NLP задачах. Все действия пользователя перед подачей на вход модели one-hot кодируются. После подачи каждого действия каждый скрытый слой LSTM выдает вектор, который выражает текущее состояние сети в пространстве малой размерности. Выходы последнего слоя собираются в одну матрицу, которая затем приводится к фиксированному размеру с помощью отбрасывания лишних векторов в случае длинных последовательностей действий и заполнения нулями в случае коротких последовательностей. Полученная матрица подается на вход CNN сети, которая в свою очередь предсказывает аномальность поведения. Авторы пишут, что их метод показал $AUC ROC=0.9449$ на отложенной выборке CERT.\\

В другой работе \cite{saaudiInsiderThreatsDetection2019} используется обратный подход. В ней CNN с одномерными сверточными слоями сначала пытается извлечь признаки, затем они подаются на вход LSTM для классификации. Значимое отличие от предыдущей работы заключается в том, что отображение поведения пользователей происходит с помощью представления каждого отдельного действия вектором малой размерностью. Значения этого вектора характеризуются соседними действиями, которые обычно встречаются вместе с ним. Авторы не указывают точно какой метод они использовали, но по описанию это очень похоже на популярный подход word2vec \cite{mikolovEfficientEstimationWord2013a}. Также авторы использовали технику SMOTE \cite{chawlaSMOTESyntheticMinority2002} для семплирования объектов малого класса и исправления сильного дисбаланса классов в наборе данных. SMOTE генерирует синтетические данные, которые похожи на $k$ ближайших соседей малого класса.\\

Обе работы \cite{saaudiInsiderThreatsDetection2019} и \cite{yuanInsiderThreatDetection2018b} ставят свою задачу как бинарную классификацию, в которой алгоритму необходимо определить аномальность поведения пользователя за некоторый промежуток времени (в обеих статьях рассматривают данные по дням)

В работе \cite{yuanAttentionBasedLSTMInsider2019} исследуется применение механизма \textbf{Attention} (внимание) для задачи обнаружения инсайдеров. Этот механизм позволяет сети обращать особое внимание для некоторых важных действий. Attention-слой в данной сети собирают выходы LSTM-слоя в один вектор, присваивая различный вес каждому выходу в зависимости от его важности. Вычисления в Attention-слоя следующие: \\
\begin{align}
\mathbf{u_t} = &\: \tanh (\mathbf{W_a}\mathbf{h_t}+\mathbf{b_a})\\
\alpha_t = &\: \frac{\exp (\mathbf{u_t^T}\mathbf{u_a})}{\sum_i \exp (\mathbf{u_i^T}\mathbf{u_a}}\\
\mathbf{v} = &\: \sum_t \alpha_t\mathbf{h_t}
\end{align}
где $\mathbf{W_a}$ - обучаемая матрица параметров, $\mathbf{u_a}$ - обучаемый вектор контекста, $\mathbf{h_t}$ t-ый выход из LSTM-слоя.\\
Полученный вектор $v$ подается на вход Softmax классификатору:
\begin{equation}
p(\hat{y}=k | \mathbf{v}) = \frac{\exp(\mathbf{W_k^T}\mathbf{v} + \mathbf{b_k})}{\sum^K_{i=1} \exp(\mathbf{W_i^T}\mathbf{v} + \mathbf{b_i})}
\end{equation}
где, $\hat{y}$ - предсказанный класс, $K$ - количество классов, $\mathbf{W_k}$ и $\mathbf{b_k}$ - обучаемые параметры для k-ого класса.\\
Как показали эксперименты в \cite{yuanAttentionBasedLSTMInsider2019}, добавление Attention увеличивает AUC ROC для LSTM и RNN сетей при использовании набора данных CERT.\\

Для поиска аномалий также возможно применение нейронных сетей из области распознавания изображений. Вдохновленные недавними успехами в применении нейросетей для анализа изображений для классификации вредоносных программ, \cite{gImageBasedFeatureRepresentation2019} применили этот подход для задачи обнаружении инсайдерской угрозы. В этой работе из набора данных CERT было вручную отобрано 20 признаков, и для каждого пользователя отдельно по этим признаком составлялись изображения 32 на 32 пикселей, которые подавались предобученной на ImageNet популярным нейросетевым моделям VGG16 и MobileNet. В результате было получено 99.32 точность и полнота на отложенной выборке.\\

В \cite{leggAutomatedInsiderThreat2017} строится трехуровневая система для обнаружения инсайдерских угроз. Первый уровень предполагает собой профилирование поведения пользователей и целых ролей в совокупности. На этом уровне срабатывает тревога при нарушении заданных правил.\\
 На следующем уровне для каждого пользователя собирается матрица дневных наблюдений, размерность которой уменьшается с помощью PCA. Последняя строка матрицы означает поведение пользователя в текущий день. В качестве оценки аномальности берётся расстояние последней строки до начала координат в новом пространстве. Вычисляется сразу множество оценок аномальности, каждая из которых соответствует собственному подмножеству используемых признаков и различному набору весов признаков. Тревога на этом уровне срабатывает, если какая-либо из оценок аномальности превышает заданные пороги.\\
 На третьем уровне анализируются полученные оценки аномальности, чтобы понять, насколько эти оценки необычны в сравнении с обычным поведением и, при необходимости, дать тревогу. Авторы предлагают для этого сразу множество вариантов.\\
Следует отметить, что в данной работе также учитывается контент файлов, но лишь как дополнительный компонент системы. Авторы предлагают использовать классический метод ``мешок слов'' или технику Linguistic Inquiry and Word Count (\textbf{LIWC}) \cite{tausczikPsychologicalMeaningWords2010}, которая заключается в распределение слов по категориям, имеющие смысл с точки зрения психологии.\\
В работе для обучения CERT, без указания точной версии, а для валидации использовался их собственный синтетический набор данных, похожий по структуре на CERT.
\\

% Несмотря на то, что набор данных CERT кроме информации о совершенных действиях пользователя также содержит сгенерированный контент файлов, писем и веб-сайтов, в подавляющем большинстве работ контент файлов не анализируется или используется лишь как дополнительный модуль, как например в \cite{leggAutomatedInsiderThreat2017}, где контент анализировался с помощью техники "мешок слов". Это обусловлено тем, что в наборе данных CERT до пятой версии генерации происходила с помощью ``мешка слов'' по смеси тем пользователя. То есть контент представляет собой неупорядоченный набор ключевых слов, которые которые представляют темы, интересующие пользователя. Поэтому использовать техники, учитывающие порядок слов, как например N-граммы, не имеет смысла.\\

% Однако последние работы в сфере NLP сделали большой прогресс в получении ``осмысленного'' признакового представления текстовых данных основанные на семантическом значении слов. Техники word2Vec\cite{mikolovEfficientEstimationWord2013a}, GloVe и ELMO позволяющие получать признаковое описание слов в пространстве малой размерности, позволили улучшить качество моделей для большинства NLP задач. Также в конце 2018 года появилась модель BERT \cite{devlinBERTPretrainingDeep2019}, которая улучшила качество работы по 6 различным NLP задачам. Обученный BERT также способен выдавать признаковое представление отдельных слов. Также в \cite{noeverClassifierSuitesInsider2019} отмечалось, сентимент-анализ содержимого писем и вебсайтов оказывает значительное влияние на качество моделей.\\

Также, как показано в статье \cite{yuanAttentionBasedLSTMInsider2019}, техника attention в применении к поведенческой информации пользователя, способна значительно улучшить качество работы модели, позволяя модели давать разный вес элементам последовательности в зависимости от важности этого элемента. Это открывает потенциал для работы с семейством моделей BERT \cite{devlinBERTPretrainingDeep2019}, чей успех в NLP задачах приписывается слоям-трансформерам, которые, в свою очередь, используют attention. Поэтому с помощью модели BERT можно анализировать последовательность действий пользователя.\\

В \cite{leEvaluatingInsiderThreat2018} представлен подход представления данных для последющего анализа, который заключаются в использовании как _последовательных_ данных (последовательность действий конкретного пользователя за некоторый период времени), так и _численных_ данных. Численные данные в данной работе делятся на пользовательские и данные о действиях. Пользовательские данные в наборе данных CERT представлены информацией о роли пользователя в подразделении, его отделе и психометрике. Данные о действиях получаются подсчетом количества совершенных действий одной категории за рассматриваемый промежуток времени. Контентные данные набора данных CERT не были рассмотрены по силу синтетической природы набора данных.

Архитектуры рекуррентных сетей изначально предполагают работу только с последовательными данными. Чтобы совместить последовательные и численные данные можно использовать подход предложенный в \cite{karpathyDeepVisualSemanticAlignments2015}. В данной статье решалась задача автоматической аннотации изображений, которая состоит в выделением некоторых участков изображения с различными объектами и генерация текстового описания объеков в этих участках. Для генерации текста использовалась архитектура RNN, в которой начальное состояние скрытых слоев зависит от информационного вектора о соответствующем участке изображения. Таким образом, все состояния рекуррентной сети обусловлены содержанием изображения. Этот подход успешно решал поставленную задачу.
Нетрудно расширить этот подход для произвольного числового вектора $\overrightarrow{x}$, который также называется _условным_. Мы можем преобразовать условный вектор так, чтобы он совпадал по размеру с вектором скрытого состояния рекуррентной сети $\overrightarrow{h}$. Для этого достаточно применить следующее простое преобразование:

$$\overrightarrow(h_0) = W\overrightarrow(x) + \overrightarrow(b)$$

,где $W$ и $\overrightarrow(b)$ являются обучаемыми параметрами. Затем, полученным вектором можно инициализировать скрытое состояние рекуррентой сети. Процесс повторяется для каждой обрабатываемой рекуррентной сетью последовательностью.

В статье \cite{leAnalyzingDataGranularity2020b} на наборе данных сравниваются модели с различными рассматривамыми периодами для каждого пользователя: день, неделя и пользовательская сессия. Для данного периода собирались числовые данные о частоте событий и их статистические признаки, такие как среднее, медиана и стандартное отклонение. В результате исследования модели, которые были обучены на полных пользовательских сессиях дали лучший результат.

LSTM также используется в Unsupervised-режиме в работе \cite{paulLACLSTMAUTOENCODER}. В данной работе рассматривается обучение LSTM-автокодировщика. Также авторы в данной работе разбивают всех пользователей в наборе данных CERT по 8 непересекающимся сообществам. Авторы предлагают находить потенциальных инсайдеров по тому, насколько они выделяются внутри своего сообщества. Эксперименты показали, что таким способом можно успешно найти всех инсайдеров среди пяти наиболее аномальных пользователей из каждого сообщества. Эксперименты проводились на наборе данных CERT v6.2.

В работе \cite{tuorDeepLearningUnsupervised2017} используется похожий Unsupervised-подход с LSTM. Модель была обучена на задаче предсказания следующего действия в последовательности. Для определения аномальности используется отклонения действий пользователя от предсказанных моделью. Авторы называют главными преумеществами своего подхода интерпретируемость результатов модели и возможность онлайн-обучения.

\section{Выводы по обзору}

Обзор литературы по существующим решениям показал, что тема обнаружения инсайдерских угроз является актуальной, и в данный момент привлекает большой интерес исследователей. Синтетический набор данных CERT является текущим де-фактом стандартом для обучения и валидации моделей. По обзору моделей становится понятно, что тренд на популярность нейросетевых моделей не обошел и задачу обнаружения инсайдерских угроз. Существуют способы, которые ставят задачу как поиск аномалий (метод обучения без учителя) и как классическую классификацию (метод обучения с учителем). Исходя из изученных работ, задача бинарной классификации, основанная на наборе данных CERT, является наиболее актуальной. Неформально эта задача представляет собой определить аномальность поведения пользователя за некоторый промежуток времени (день, неделя или пользовательская сессия).
