
% \chapter{Аннотация}

% Целью данной работы является изучение средств для создания системы автоматического обнаружения инсайдерских угроз по поведенческой информации пользователей.\\
% Рассмотрен ряд различных существующих решений, которые основаны на синтетическом наборе данных инсайдерских угроз CMU CERT. На основе анализа моделей, для разработки выбрана архитектура нейросетевой модели с LSTM-кодировщиком поведения и CNN-классификатора. Результаты нейросетевых моделей сравниваются с классическими алгоритмами машинного обучения.\\
% В процессе работы были испробованы различные техники и изменения для улучшения качества предсказаний моделей такие как применение embedding-слоя в LSTM-кодировщике, взвешенная функция потерь, использование Batch Normalization, добавление дополнительных признаков, в том числе признаков о содержимом файлов.

\chapter{Введение}

\textbf{Внутренние (инсайдерские) вторжения} — это вредоносные для организации угрозы, которые исходят от людей внутри организации, таких как работники, бывшие работники, подрядчики или деловые партнеры, имеющих доступ к информации о методах безопасности внутри организации, данных и компьютерных системах.\\

Инсайдерские угрозы в данный момент является серьезной проблемой для многих организаций. Их обнаружение является очень трудной задачей в силу того, что инсайдеры для злонамеренных действий используют доверенный им доступ, из-за чего они могут легко обходить системы внешней защиты информации.\\

Существует ряд технологий, для решения проблем инсайдерских угроз:\\
\begin{itemize}
\item DLP (Data Loss Prevention) – предотвращение утечек с помощью анализа потоков данных, пересекающих периметр информационной системы\\
\item SIEM (Security Information and Event Management) – анализ в реальном времени событий безопасности\\
\item IAM (Identity and Access Management) – управление учетными данными пользователей, системами контроля и управления доступом\\
\item UEBA (User and Entity Behavior Analytics) – анализ действий пользователя и принятие решения на основе исторических данных\\
\end{itemize}

DLP, SIEM и IAM системы способны бороться с инсайдерами только на самом последнем этапе утечки данных, когда инсайдер пытается провести эксфильтрацию данных. При этом, до этого момента инсайдер проходит стадии подготовки утечки, которая может длиться месяцами и в течение этого этапа инсайдер проявляет аномальное поведение. Поэтому существует отдельный класс решений UEBA, который направлен на анализ поведения пользователей и способны обнаруживать ранние признаки утечки информации.\\

Под поведенческой информацией понимается совокупность данных, которые описывают:\\
\begin{itemize}
\item Контекст – структурированные данные, описывающие атрибуты операций, которые пользователь выполняет с документами;
\item Контент – неструктурированные данные в виде содержимого документов, ассоциированные с операциями пользователя.
\end{itemize}

\section*{Актуальность задачи}

По отчетам Ponemon за 2020 год \cite{PonemonReport20182018} стоимость ущерба от инсайдерских угроз составляет 11,45 миллионов долларов в среднем за атаку. При этом средняя стоимость атак выросла на 31\% в два раза с 2018 года. По тому же отчету, частота инсайдерских инцидентов также заметно выросла. Также приводится, что из всех методов борьбы с инсайдерской угрозой UEBA системы лидируют по степени сокращения финансового ущерба. В среднем они сокращают 3,42 миллиона долларов. Это как раз связано с тем, что UEBA системы способны обнаруживать аномальное поведение пользователя на ранних этапах утечки информации. Это подтверждается в \cite{2019InsiderThreat}, где подчеркивается, что в силу того, что инсайдеры имеют мало препятствий, время с первых несанкционированных действий до обнаружения часто занимает месяцы и годы.\\ \\

Согласно McKinsey \cite{InsiderThreatHuman} из всех опубликованных в VERIS Community Database публичных случаях утечки информации за период с 2012 по 2017 год, в 50\% присутствовал элемент инсайдерской угрозы. При этом, 38\% случаев из этих наличествует злонамеренные действия инсайдеров. В отчете Verizon за 2019 год \cite{2019InsiderThreat} приводится, что 57\% взломов базы данных организации включают в себя инсайдерские угрозы.\\

Отчеты Haystax \cite{veriatoInsiderThreatReport}\cite{companyInsiderThreatReport} и Gartner \cite{EmergingInsiderThreat2018} подтверждают растущие опасения организаций по этому поводу и тренд на рост частоты атак инсайдеров.

Согласно отчету Gartner \cite{GartnerReportMarket2019} рынок самостоятельных UEBA-систем умирает, но в данный момент очень востребованы SIEM-системы, которые включают в себя UEBA.\\

По приведенным выше отчетам становится очевидной актуальность изучения и разработки моделей машинного обучения, способные определять ранние признаки аномального поведения корпоративных пользователей.

Существующие UEBA-решения нацелены в первую очередь на анализ структурированной контекстной информации о поведении пользователя которые включают в себя обычно данные об операциях с файлами, электронной почты, подключении новых устройств и данные из системного журнала ОС. Это объясняется тем, что анализ контентной информации намного более затруднителен из-за своей неструктурированности и очень большого объёма. В то же время, анализ контента позволяет выявлять случаи, при которых поведения пользователя остаётся таким же, но меняется содержимое файлов, с которыми он работает. Анализ только структурированной информации такие случаи выявить не может, что позволяет утверждать о ценности контента для рассматриваемой задачи. Актуальность анализа текстовых данных также подвтерждается отчётом Gartner \cite{GartnerReportMarket2019}.\\


\chapter{Постановка задачи}

\begin{enumerate}
	\item Реализация системы сбора поведенческой информации пользователя. Типы поведенческой информации:
	\begin{enumerate}
		\item контекст пользовательских операций;
		\item контент файлов;
	\end{enumerate}
	\item Исследование и разработка методов обнаружения аномального поведения пользователей на основе собираемой поведенческой информации.
\end{enumerate}